\documentclass{../template}
\renewcommand{\studentOne}{Daniel Name, 123456}
\renewcommand{\studentTwo}{Zweiter Name, 654321}
\renewcommand{\studentThree}{Dritter Name, 162534}
\renewcommand{\tutorium}{99}

\setuebungsnr{0}

\begin{document}
	

	
	% Hier geht die eigentliche Lösung der Aufgaben los
	
	\section*{Aufgabe 1}
	
	Das \LaTeX{}-Kompendium auf Wikibooks\footnote{\url{https://de.wikibooks.org/wiki/LaTeX-Kompendium}} ist eine gute Einführung.
	Für die Zusammenarbeit bietet sich das RWTH Gitlab an\footnote{\url{https://git.rwth-aachen.de}}.
	
	
	\section*{Aufgabe 2}
	
	Als \LaTeX{}-Editor bzw. -IDE ist TeXStudio \footnote{\url{https://www.texstudio.org/}} ein guter Kandidat.
	

	
	\section*{Aufgabe 3}
	
		Für Teilaufgaben kann man Aufzählungen verwenden:
	\begin{itemize}
		\item als Aufzählung
		\item ohne Nummerierung,
	\end{itemize}
	oder auch
	\begin{enumerate}
		\item mit
		\item Nummerierung.
	\end{enumerate}
	
	\section*{Aufgabe 4}
	
	Kleine Diagramme lassen sich ebenfalls erstellen, zum Beispiel:
	
	\begin{center}
		\begin{tikzpicture}
		\node (1) {$v_1$};
		\node [right of=1] (2) {$v_2$};
		\node [below of=2] (3) {$v_3$};
		\draw (1) -- (2);
		\draw (1) -- (3);
		\end{tikzpicture}
	\end{center}
	
	oder auch etwas komplizierter:
	
	\begin{center}
		% Stil festlegen, der dann für alle Knoten verwendet wird
		\tikzset{dot/.style={circle, draw=black, fill=black, inner sep=0pt, minimum size=5pt}}
		% Das eigentlich Diagramm
		\begin{tikzpicture}[node distance=1.5cm]
		\node [dot,label={left:$v$}] (1) {};
		\node [dot,right of=1] (2) {};
		\node [dot,below of=2,label={right:$w$}] (3) {};
		\node [dot,below of=1] (4) {};
		\path [draw=black, ->, >=stealth', shorten <=2pt, shorten >=2pt] % Optionen nur für das Aussehen
		(1) edge (3)
		(1) edge (4)
		(2) edge (3)
		(4) edge [loop left] ()
		(1) edge node [above] {$e$} (2);
		\end{tikzpicture}
	\end{center}
	
	Alternativ können auch externe Dateien eingebunden werden (z.B.\ Bildformate, PDF).
	% Dazu ganz oben das Paket graphicx einbinden: \usepackage{graphicx}
	% und an der entsprechenden Stelle dann die Grafik laden: \includegraphics{datei}
	
\end{document}

